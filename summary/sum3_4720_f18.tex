\documentclass[twoside]{article}
\usepackage{amsgen,amsmath,amstext,amsbsy,amsopn,amssymb,hyperref}
\usepackage{graphicx}
\usepackage{epsfig}

\setlength{\oddsidemargin}{0.1 in} \setlength{\evensidemargin}{-0.1
in} \setlength{\topmargin}{-0.6 in} \setlength{\textwidth}{6.5 in}
\setlength{\textheight}{10.5 in} \setlength{\headsep}{0.1 in}
\setlength{\parindent}{0 in} \setlength{\parskip}{0.1 in}

\newcommand{\homework}[2]{
   \pagestyle{myheadings}
   \thispagestyle{plain}
   \newpage
   \setcounter{page}{1}
   \noindent
   \begin{center}
   \framebox{
      \vbox{\vspace{2mm}
       \hbox to 6.28in { {\bf Math 4720:~Statistical Methods \hfill} }
       \vspace{6mm}
       \hbox to 6.28in { {\Large \hfill #1 (#2)  \hfill} }
       \vspace{6mm}
      \vspace{2mm}}
   }
   \end{center}
   \markboth{#1}{#1}
   \vspace*{4mm}
}

\newcommand{\bbF}{\mathbb{F}}
\newcommand{\bbX}{\mathbb{X}}
\newcommand{\bI}{\mathbf{I}}
\newcommand{\bX}{\mathbf{X}}
\newcommand{\bY}{\mathbf{Y}}
\newcommand{\bepsilon}{\boldsymbol{\epsilon}}
\newcommand{\balpha}{\boldsymbol{\alpha}}
\newcommand{\bbeta}{\boldsymbol{\beta}}
\newcommand{\0}{\mathbf{0}}

\begin{document}

\homework{$3^{rd}$ Week Summary}{09/17/18}\vspace{-.35in}
\begin{itemize}
\item The \textbf{probability} of any event A of a random phenomenon is the proportion of times the event would occur in a very long series of repetitions.
\item The \textbf{sample space $S$} of a random phenomenon is the set of all possible outcomes.
\item An \textbf{event} is a subset of the sample space.
\item Probability Rules
\subitem The probability $P(A)$ of any event $A$ satisfies $0 < P(A) < 1$.
\subitem If $S$ is the sample space, then $P(S) = 1$.
\subitem Two events $A$ and $B$ are \textbf{disjoint} if they have no outcomes in common and so can never occur together. If $A$ and $B$ are disjoint, $P(A \ \textrm{or} \ B) = P(A \bigcup B) = P(A) + P(B)$
\subitem For any event $A$, $P(A$ does not occur$) = P(\overline{A}) = 1 - P(A)$.
\item Addition rule in general : $P(A \ \textrm{or} \ B) = P(A \bigcup B) = P(A) + P(B) - P( A \bigcap B )$

\begin{figure}[h]
\begin{center}
\includegraphics[angle=0,width=6in] {venn_tree.jpg}
\end{center}
\end{figure}

\item $P(A|B)$, the \textbf{conditional probability} of A given that B has occurred, can be thought of an adjusted version of the probability of A in light of the additional information that B has occurred.
\subitem When $P(B) > 0$, the conditional probability of A given B is: $P(A|B) = \dfrac{P(A \ \textrm{and} \ B)}{P(B)} = \dfrac{P(A \bigcap B)}{P(B)}$

\item General \textbf{multiplication} rule : $P(A \ \textrm{and} \ B) = P(A \bigcap B) = P(A|B) P(B) = P(B|A) P(A)$.
\item Two events $A$ and $B$ that both have positive probability are \textbf{independent} if: $P(A|B) = P(A)$ or $P(A \bigcap B)=P(A)P(B)$.
\item Bayes' Rule : $P(A|B)=\dfrac{P(B|A)P(A)}{P(B)}=\dfrac{P(B|A)P(A)}{P(B|A)P(A)+P(B|\overline{A})P(\overline{A})}$

\item Law of total probability: If $A_1, A_2, \ldots, A_k$ are disjoint events whose probabilities are not $0$ and add to exactly 1, then:
    $$P(B)={P(B|A_1)P(A_1)+\ldots+P(B|A_k)P(A_k)}$$
\item General Bayes' Rule : Suppose that $A_1, A_2, \ldots, A_k$ are disjoint events whose probabilities are not $0$ and add to exactly 1, then :
    $$P(A|B)=\dfrac{P(B|A_i)P(A_i)}{P(B|A_1)P(A_1)+\ldots+P(B|A_k)P(A_k)}$$

\end{itemize}

\end{document}
